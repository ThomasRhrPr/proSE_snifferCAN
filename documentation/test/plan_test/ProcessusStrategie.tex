% Auteurs : Camille Constant, Paul Trémoureux

\section{Processus et Stratégie de test}
\label{sec:process}

\subsection{Objectifs (actions de test)}
\label{sec:process:objectifs}

\noindent Exigence : exigence concernée\\
Risque : risque concerné (cf. section \ref{sec:peri:risques})\\
Niveau : niveau de test (S : Système, I : Intégration, U : Unitaire)\\
Technique : technique de test (AP : Analyse Partitionnelle ou Classes d'équivalence, AL : Analyse aux limites, CU : Cas d'Utilisation, PC : Protocole de Communication)\\

\noindent\begin{longtable}[c]{|p{0.5cm}|p{2.8cm}|p{1.7cm}|p{1.2cm}|p{1.2cm}|p{1.8cm}|p{3.6cm}|}
\hline
\bf \centering N\textdegree & \bf Énoncé de l'objectif & \bf Exigence & \bf Risque & \bf Niveau & \bf Technique & \bf Conditions de mesure/niveau d'atteinte prévu.\\
\hline
\endhead
\centering 1 & Tester la communication & Fonct. P0 & RPRD1 & I, S & AP, PC, CU & Fonctionnalités P0/100\% des fonctionnalités testées en utilisant les classes d'équivalence, le protocole de communication et les cas d'utilisation.\\
\hline
\centering 2 & Smoke test & Toutes & RPRJ1, RPRJ2, RPRJ4 & S & Test par expérience & Nombre d'anomalies bloquantes/pas d'anomalie bloquante.\\
\hline
\centering 3 & Tester 100\% des fonctionnalités P0 & Fonct. P0 & RPRJ3, RPRD1 & S & AP, CU & Fonctionnalités P0/100\% des fonctionnalités testées en utilisant les classes d'équivalence et les cas d'utilisation.\\
\hline
\centering 4 & Tester 100\% des fonctionnalités non P0 & Fonct. non P0 & RPRJ3, RPRD1 & S & AP, CU & Nombre de cas d'utilisation/tous les cas d'utilisation testés.\\
\hline
\centering 5 & Contrôle qualité des productions. & Toutes & RPRJ6 & U & Relecture, PAQL & Respect des règles du PAQL.\\
\hline
\end{longtable}

\subsection{Organisation}
\label{sec:process:orga}

\subsubsection{Découpage en phase de tests/campagnes}
\label{sec:process:orga:decoupage}

Deux campagnes de tests système sont prévues dans le projet pour chaque lot/incrément :
\begin{itemize}
    \item Campagne de tests système comprenant les tests fonctionnels et non fonctionnels pour atteindre les différents objectifs de tests ci-dessus ;
    \item Campagne de retest (vérification de la correction des anomalies détectées) et de régression.
\end{itemize}

\subsubsection{Gestion des rapports d'anomalie}
\label{sec:process:orga:anomalies}

Les anomalies sont gérées dans l'ENTP sous forme de tâche.
Dès l'observation d'une défaillance dans le produit, un rapport d'anomalie est rédigé dans l'ENTP.

\subsection{Critères d'acceptation des tests}
\label{sec:process:accept}

Pour le passage en test de validation système, la phase de smoke test ne doit pas détecter d'anomalie bloquante.\\
Pour la mise en production, aucune anomalie bloquante ni majeure n'est acceptée.

\noindent\begin{longtable}[c]{|p{.25\textwidth}|p{.75\textwidth}|}
\hline
{\bf Anomalie bloquante} & La fonctionnalité n'est pas utilisable.\\
\hline
{\bf Anomalie majeure} & La fonctionnalité ne répond pas à ses exigences mais une solution de contournement existe pour utiliser la fonctionnalité, ou la fonctionnalité est utilisable en l'état (par exemple, anomalie dans une règle de calcul).\\
\hline
{\bf Anomalie mineure} & La fonctionnalité est utilisable mais pas de façon optimale (par exemple, problème d'ergonomie ou de charte graphique).\\
\hline
\end{longtable}

\subsection{Critères d'arrêt}
\label{sec:process:arret}

Les tests d'une fonctionnalité s'arrêteront si une anomalie bloquante est découverte ne permettant pas de poursuivre les tests de cette fonctionnalité.

\subsection{Activités de test}
\label{sec:process:activites}  
  
L'activité de test sera faite par {\equipe} tout au long du cycle de développement, via notamment :
\begin{itemize}
    \item Des tests de validation sur le comportement nominal et aux limites du système ;
    \item Des tests d'intégration sur le comportement nominal du programme {\appliC} ;
    %Ligne à ajouter au second incrément, ne pas oublier les modifications de ponctuation.
    \item Des tests unitaires nominaux sur certaines classes de l'application {\appliA} et sur certains modules du programme {\appliC}.
\end{itemize}

\subsubsection{Planification}

La planification des tests système est réalisée par {\equipe}.

\subsubsection{Conception}

La conception des tests système est réalisée par {\equipe}.
La conception des jeux de données de test est réalisée par {\equipe}.

\subsubsection{Exécution}

L'exécution des tests système est réalisée par {\equipe}.
L'exécution des tests d'acceptation est réalisée par {\client}.

\subsubsection{Bilan}

{\equipe} rédige un bilan de test en fin de campagne de test système.

\subsection{Documents de test et livrables}
\label{sec:process:livrables}

\noindent\begin{longtable}[c]{|l|c|}
\hline
{\bf Documentation} & {\bf Livrable à transmettre} \\
\hline
Plan de test & X \\
\hline
Dossier de test & X \\
\hline
Rapport de test & X \\
\hline
Matrice de conformité exigences et tests & X \\
\hline
Rapport d’anomalies & X \\
\hline
{\bf Données} & \\
\hline
Documents d'analyses partitionnelles et aux limites & \\
\hline
Jeux de données de tests & X\\
\hline
{\bf Automatisation des tests} & \\
\hline
Code de test & \\
\hline
\end{longtable}