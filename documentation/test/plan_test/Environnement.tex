% Auteurs : Camille Constant, Paul Trémoureux 

\section{Environnement de test}
\label{sec:env}

\subsection{Environnement matériel et logiciel de test}
\label{sec:env:env}

Liste des matériels et logiciels nécessaires à l'exécution des tests :
\begin{itemize}
    \item E\_PC : Linux Ubuntu (64-bit) ;
    \item E\_Smartphone : Samsung Galaxy A20e sous système Android 9 ;
    \item E\_Raspberry : pi 3B+ ;
    \item E\_Passerelle : RS485 ;
    \item E\_ICSim : version du 12/06/2020 ;
    \item E\_PEAK : PEAK PCAN.USB IPEH-002021 175459 ;
    \item E\_Banc\_De\_Test : KEREVAL Banc de Test.
\end{itemize}

\subsection{Outils de test}
\label{sec:env:outils}

Les tests seront au maximum automatisés grâce aux outils suivants :
\begin{itemize}
    \item Tests unitaires : Framework de test Android (basé sur JUnit), bouchonnage Mockito, CMocka ;
    \item Tests d'intégration : JMeter ;
    \item Tests de validation : automatisation avec Robot Framework, sinon tests manuels ;
    \item Dossier de test : Squash TM avec intégration de la gestion d'anomalies via Redmine.
\end{itemize}
