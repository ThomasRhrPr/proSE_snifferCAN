\paragraph{La classe ProxyLogger}

\begin{minipage}
    {\linewidth}
    \centering
    \includegraphics[width=0.65\linewidth]{../schemas/Conception_detaillee/classe_ProxyLogger.pdf}
    \captionof{figure}{Diagramme de classe de ProxyLogger}
\end{minipage}

\subparagraph{Philosophie de conception \newline} 

\medspace

La classe ProxyLogger qui est dans le programme {\nomLogiciel} a pour rôle de simuler le comportement de la classe Logger présente dans l'application {\nomApplication}.

\subparagraph{Description structurelle \newline}

\medspace

\textbf{Attributs :}

N.A.

\textbf{Services offerts :}

\begin{itemize}
    \item \textbf{setFrame(frame : Frame) : ErrorState\_e} --- Opération qui permet de fournir les trames sniffées au Logger.
    \item \textbf{encodeMessage(frame : Frame, message : String) : ErrorState\_e} --- Opération qui permet d'encoder une trame en une chaîne de caractères.
\end{itemize}
