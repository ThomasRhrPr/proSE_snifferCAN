\paragraph{Exemples}
\label{sec:exemple-CANdroidToCANgateway}

Pour rappel, nous appliquons un protocole textuel, les données envoyées sont sous la forme de chaîne de caractères.

\subparagraph{Envoyer des trames \newline}

\medspace

Les échanges présentés dans le diagramme de la \autoref{fig:inter-CU_Envoyer_des_trames_-_Scénario_nominal} représente un envoi de donnée de l'application {\nomApplication} vers le programme {\nomLogiciel} via le protocole de communication décrit dans le \autoref{sec:formalisation-protocole-CANdroidToCANgateway}. Cette donnée correspond à une séquence de trame et peut être de la forme suivante :\\

\begin{minipage}
    \textwidth
    \centering
    \begin{tabular}{|c|c|c|}
        \hline
        nb=2\textbackslash n & \#19B\$6@00000E000000/1000\textbackslash n & \#19B\$6@000007000000/0\textbackslash n \\
        \hline
    \end{tabular}
\end{minipage}

\medspace

La première trame est envoyée toutes les secondes, et permet d'ouvrir la portière de la porte avant-gauche sur le Simulateur ICSim. La seconde trame est envoyée une seule fois et permet d'ouvrir la portière arrière droite sur le Simulateur ICSim.

Ensuite, une seconde donnée est envoyée, correspondant à une demande d'état du Mode Envoi. Elle est de la forme suivante :\\

\begin{minipage}
    \textwidth
    \centering
    \begin{tabular}{|c|}
        \hline
        ? \textbackslash n\\
        \hline
    \end{tabular}
\end{minipage}

\medspace

\subparagraph{Arrêter d'envoyer des trames \newline}

\medspace

Les échanges présentés dans le diagramme de la \autoref{fig:inter-CU_Arrêter_envoi_des_trames_-_Scénario_nominal} représente un envoi de données de l'application {\nomApplication} vers le programme {\nomLogiciel} via le protocole de communication décrit dans le \autoref{sec:formalisation-protocole-CANdroidToCANgateway}. Ces données correspondent à une demande d'arrêt d'envoi de trames, puis une demande d'état du mode Envoi. Elles sont de la forme suivante  :\\

\begin{minipage}
    \textwidth
    \centering
    \begin{tabular}{|c|}
        \hline
        !\textbackslash n\\
        \hline
    \end{tabular}
\end{minipage}

\medspace

Puis ce message :

\medspace

\begin{minipage}
    \textwidth
    \centering
    \begin{tabular}{|c|}
        \hline
        ?\textbackslash n\\
        \hline
    \end{tabular}
\end{minipage}