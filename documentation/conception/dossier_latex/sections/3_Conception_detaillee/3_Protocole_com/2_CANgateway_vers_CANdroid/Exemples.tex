\paragraph{Exemples}
\label{sec:exemple-CANgatewayToCANdroid}

Pour rappel, nous appliquons un protocole textuel, les données envoyées sont sous la forme de chaîne de caractères.


\subparagraph{Recevoir des trames \newline}

\medspace

Les échanges présentés dans le diagramme de la \autoref{fig:inter-CU_Recevoir_des_trames_-_Scénario_nominal} représente un envoi de donnée du programme {\nomLogiciel} vers l'application {\nomApplication} via le protocole de communication décrit dans le \autoref{sec:formalisation-protocole-CANgatewayToCANdroid}. Cette donnée correspond à une trame et peut être de la forme suivante :\\

\begin{minipage}
    \textwidth
    \centering
    \begin{tabular}{|c|}
        \hline
        \textasciitilde{}frame\#19B\$6@00000E000000\textbackslash n\\
        \hline
    \end{tabular}
\end{minipage}

\medspace

Cette trame permet d'ouvrir la porte avant-gauche de la portière sur le Simulateur ICSim.


\subparagraph{Envoyer des trames \newline}

\medspace

Les échanges présentés dans le diagramme de la \autoref{fig:inter-CU_Envoyer_des_trames_-_Scénario_nominal} représente un envoi de donnée du programme {\nomLogiciel} vers l'application {\nomApplication} via le protocole de communication décrit dans le \autoref{sec:formalisation-protocole-CANgatewayToCANdroid}. Il y a deux données transmises lors d'un envoi de trames :
\begin{itemize}
    \item L'état du Mode Envoi ;
    \item La trame envoyée.
\end{itemize} 
L'état du Mode Envoi est de la forme suivante :\\

\begin{minipage}
    \textwidth
    \centering
    \begin{tabular}{|c|}
        \hline
        \textasciitilde{}sendingState\#1\textbackslash n\\
        \hline
    \end{tabular}
\end{minipage}

\medspace

La trame envoyée est de la forme suivante :\\

\begin{minipage}
    \textwidth
    \centering
    \begin{tabular}{|c|}
        \hline
        \textasciitilde{}frame\#19B\$6@00000E000000\textbackslash n\\
        \hline
    \end{tabular}
\end{minipage}

\medspace

\subparagraph{Arrêter d'envoyer des trames \newline}

\medspace

Les échanges présentés dans le diagramme de la \autoref{fig:inter-CU_Arrêter_envoi_des_trames_-_Scénario_nominal} représente un envoi de donnée du programme {\nomLogiciel} vers l'application {\nomApplication} via le protole de communication décrit dans le \autoref{sec:formalisation-protocole-CANgatewayToCANdroid}. Cette donnée correspond à un état du Mode Envoi et est de la forme suivante :\\

\begin{minipage}
    \textwidth
    \centering
    \begin{tabular}{|c|}
        \hline
        \textasciitilde{}sendingState\#0\textbackslash n\\
        \hline
    \end{tabular}
\end{minipage}