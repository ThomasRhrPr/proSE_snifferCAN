\\
Utilisateur demande l'ajout d'un nouvel objet sur l'application {\nomApplication}. Si le nombre d'objets existants dans l'application {\nomApplication} est inférieur à la limite maximale, PopupAjoutObjet s'affiche. Ce pop-up permet à Utilisateur de donner un nom à l'objet qu'il souhaite ajouter. Une fois le nom saisi, Utilisateur valide la demande d'ajout. Enfin, l'objet est ajouté aux objets déjà présents sur l'application {\nomApplication} et EcranPrincipal est mis à jour en conséquence. \\
Dans le cas général, Utilisateur peut choisir de laisser le nom d'objet par défaut présent dans <champNomObjet> pré-rempli. Ce nom est obtenu grâce à l'appel de l'opération "getMaxIdObject()" qui permet de donner le numéro du nom de l'objet. Par exemple, si l'opération renvoie le nombre "4" alors le Nom d'objet par défaut est "Objet\_4".\\
De plus, si Utilisateur a entré un nom déjà utilisé par un autre objet, l'application {\nomApplication} affiche PopupErreurAjoutObjet. S'il essaie d'ajouter un objet alors que le nombre d'objet maximum est atteint, l'application {\nomApplication} affiche PopupErreurNombreObjet. Ces cas de figure ne sont pas présentés sur ce scénario, mais sont explicitement décrits dans le CU "Ajouter un objet" du dossier de spécification [\hyperref[SPEC]{dossier\_de\_specification\_SPEC\_B1\_2024}].
