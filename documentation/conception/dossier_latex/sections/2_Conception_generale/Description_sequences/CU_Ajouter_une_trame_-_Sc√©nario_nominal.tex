\\
Lorsque Utilisateur ouvre le menu de l'objet, EcranPrincipal se met à jour. Utilisateur entre une nouvelle trame dans le menu de l'objet associé et l'enregistre si le nombre de trames dans cet objet est inférieur à la limite maximale autorisée, pour l'enregistrement des trames. L'application {\nomApplication} affiche PopupModeEnvoiTrame, permettant à Utilisateur de définir le Mode Envoi de la trame. Utilisateur définit et valide son choix, la trame s'enregistre et EcranPrincipal s'affiche.\\

Dans le cas général, la saisie de la trame peut être incorrecte, l'application {\nomApplication} affiche donc PopupErreurSaisieTrame, qui indique le bon format des trames à enregistrer. Utilisateur peut également décider d'ajouter une trame alors que le nombre de trame maximum est atteint, dans ce cas, l'application {\nomApplication} afficher PopupErreurNombreTrame. De plus, dans la \autoref{fig:inter-CU_Ajouter_une_trame_-_Scénario_nominal}, le Mode Envoi est défini par défaut en mode cyclique avec une périodicité par défaut de 100 ms et Utilisateur le définit en mode ponctuel. Cependant, Utilisateur peut choisir de ne pas effectuer ce changement. Ces cas de figure ne sont pas présentés sur ce scénario, mais sont explicitement décrits dans le CU "Enregistrer une trame" du dossier de spécification [\hyperref[SPEC]{dossier\_de\_specification\_SPEC\_B1\_2024}].