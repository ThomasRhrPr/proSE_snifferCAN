\\
La première étape du scénario nominal du CU "Recevoir des trames" du dossier de spécification [\hyperref[SPEC]{dossier\_de\_specification\_SPEC\_B1\_2024}] n'est pas représentée dans le diagramme de séquence car elle est implicite. On suppose que Tableau de Bord fonctionne et est capable d'émettre des trames sur le bus CAN. \\
Périodiquement, \textit{sniffer} reçoit une trame de Tableau de Bord. Lorsqu'il reçoit une trame, il transmet celle-ci à \textit{logger} qui informe ensuite \textit{gui} de la réception d'une nouvelle trame. Cela permet ainsi à \textit{gui} de récupérer et afficher cette trame sur EcranPrincipal. \\

Le diagramme de la \autoref{fig:inter-CU_Recevoir_des_trames_-_Scénario_nominal} représente la réception d'une trame, cependant, il est possible que \textit{sniffer} reçoive beaucoup de trames en peu de temps. Dans ce cas, l'opération "notifyNewFrames()" est appelée périodiquement et \textit{gui} récupère toutes les trames reçues depuis le dernier appel de "getFrames()". Ainsi, les opérations "notifyNewFrames()" et "getFrames()" sont au pluriel car il est possible de récupérer un paquet de trames. 
Pour finir, comme on peut le voir dans la \autoref{fig:inter-CU_Interagir_avec_le_sniffer_-_Scénario_nominal}, Utilisateur a la possibilité de mettre en pause la réception des trames, ce qui suspend temporairement la boucle. \\