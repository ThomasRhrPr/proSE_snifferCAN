\\
Utilisateur sélectionne un objet et toutes les trames qui lui sont associées sont automatiquement sélectionnées en conséquence. Par la suite, L'application {\nomApplication} se met à jour. Après avoir sélectionné les éléments à supprimer, l'application {\nomApplication} affiche PopupSuppressionElement et Utilisateur valide la suppression. Une fois la suppression confirmée, l'application {\nomApplication} met à jour EcranPrincipal pour refléter les changements.\\

Dans le cas général, Utilisateur a la possibilité de sélectionner une trame sans nécessairement sélectionner l'objet qui la contient, et de la supprimer. De plus, il est possible de sélectionner plusieurs éléments simultanément pour les supprimer, et l'application {\nomApplication} met à jour EcranPrincipal en conséquence. Ces cas de figure ne sont pas présentés sur ce scénario, mais sont explicitement décrits dans le CU "Supprimer un élément" du dossier de spécification [\hyperref[SPEC]{dossier\_de\_specification\_SPEC\_B1\_2024}]. 