\\
Utilisateur peut interagir avec le sniffer et réaliser plusieurs actions : 
\begin{itemize}
    \item Revenir en haut du sniffer pour voir apparaître les nouvelles trames. 
    \item Nettoyer le sniffer c'est-à-dire supprimer toutes les trames présentes dans le sniffer. 
    \item Lancer ou mettre en pause la réception des trames. 
    \item Exporter les trames dans un fichier log en dehors de l'application {\nomApplication}. Cette action est uniquement possible si Utilisateur a mis en pause la réception des trames.
\end{itemize}
De plus, la \autoref{fig:inter-CU_Interagir_avec_le_sniffer_-_Scénario_nominal} indique un ordre spécifique pour les actions à effectuer, à savoir l'ordre du CU "Interagir avec le sniffer" du dossier de spécification [\hyperref[SPEC]{dossier\_de\_specification\_SPEC\_B1\_2024}]. Cependant, dans la version comprenant les variantes, il est indiqué que les actions peuvent être effectuées dans l'ordre souhaité, à l'exception de l'exportation des trames, qui doit être réalisée après la mise en pause du sniffer. En d'autres termes, Utilisateur a la liberté d'effectuer les actions dans l'ordre qui lui convient, excepté pour l'exportation des trames qui nécessite une mise en pause du sniffer au préalable.