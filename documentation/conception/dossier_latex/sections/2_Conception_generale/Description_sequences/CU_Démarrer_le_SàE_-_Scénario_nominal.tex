\\
Pour rappel, en dehors de la portée du système, Utilisateur met en fonctionnement Tableau de Bord, connecte la Raspberry PI au bus CAN et met la Raspberry PI sous tension.\\
Utilisateur doit ensuite démarrer manuellement le programme {\nomLogiciel} via une connexion SSH avec la Raspberry PI. Quand le programme {\nomLogiciel} est démarré, la LED de la Raspberry PI renvoie l'information à Utilisateur. Le programme {\nomLogiciel} commence alors à lire les trames du bus CAN.\\
Ensuite, Utilisateur démarre l'application {\nomApplication} et EcranPrincipal s'affiche. L'application {\nomApplication} charge l'ensemble des éléments créés lors des précédentes utilisations et les affiche sur EcranPrincipal.\\
Enfin, l'application {\nomApplication} se connecte au programme {\nomLogiciel} et EcranPrincipal se met à jour.\\
Il est possible que la connexion entre l'application {\nomApplication} et le programme {\nomLogiciel} échoue. Dans ce cas, PopupReconnexion s'affiche et demande à Utilisateur s'il souhaite réessayer. Ce scénario n'est pas représenter sur la \autoref{fig:inter-CU_Démarrer_le_SàE_-_Scénario_nominal}.\\