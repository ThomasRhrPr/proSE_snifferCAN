\\
Utilisateur ouvre le menu de l'objet et sélectionne la trame qu'il souhaite envoyer. L'application {\nomApplication} affiche ensuite EcranPrincipal. Utilisateur commence l'envoi de la trame sélectionnée.  Pour que l'envoi commence, \textit{sender} a besoin des informations concernant la trame choisie. Ces informations lui sont transmises grâce à \textit{gui} qui les récupère de \textit{dealer}. Dès que l'application passe en Mode Envoi, EcranPrincipal est mis à jour en conséquence. Afin d'afficher la trame émise dans le sniffer de l'application, il est nécessaire de transmettre la trame à \textit{logger}. Le diagramme de la \autoref{fig:inter-CU_Envoyer_des_trames_-_Scénario_nominal} ne représente que l'envoi d'une trame mais il est possible d'envoyer plusieurs trames à la fois. \\
En réalité, les opérations "askSendingState()" et "setSendingState(sendingState : SendingState)" sont appelées périodiquements pour informer \textit{gui} de l'état de l'application {\nomApplication}. \\

Dans le cas général, si Utilisateur souhaite désélectionner un élément, il lui suffit de cliquer à nouveau sur ce dernier. Si un objet a été sélectionné, il n'est pas possible de désélectionner individuellement une trame qui fait partie de ce même objet. \\
Pour chaque trame sélectionnée, \textit{gui} récupère les informations de la trame grâce à l'opération "getFrames()" de \textit{dealer}. \\
De plus, si les trames sélectionnées ont des modes d'envoi périodiques, alors l'appel des opérations "send()", "setFrame(frame)", "notifyNewFrames()", "getFrames()" et "displayMainScreen()" se répète jusqu'à la demande d'arrêt. \\
Ces cas de figure ne sont pas présentés sur ce scénario, mais sont explicitement décrits dans le CU "Envoyer des trames" du dossier de spécification [\hyperref[SPEC]{dossier\_de\_specification\_SPEC\_B1\_2024}].