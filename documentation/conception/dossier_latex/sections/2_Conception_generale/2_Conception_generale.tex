% WARNING: automatically generated file that may be overwritten or removed at any time

\section{Conception générale}

\newcommand\macroSuffix{}
\input{../animUML/Passerelle_CAN-macros}


\subsection{Architecture candidate}

\begin{figure}[H]
	\centering
	\includegraphics[max width=\textwidth,max height=.9\textheight]{../animUML/Passerelle_CAN-context}
	\caption{Architecture candidate}
	\label{fig:archiCand}
\end{figure}
Le diagramme de la \autoref{fig:archiCand} représente l'architecture candidate du système.
\\

Il s'agit de la conception générale ; l'hypothèse d'un système matériel à ressources infinies est pour l'instant posée. 

\begin{itemize}
    \item Dans ce diagramme, on retrouve deux objets représentant des interfaces homme-machine : 
    \begin{itemize}
        \item \textit{ui}, permettant de démarrer ou arrêter le programme {\nomLogiciel} et d'informer Utilisateur du bon fonctionnement du programme 
        \item \textit{gui}, permettant de démarrer et arrêter l'application {\nomApplication}. Il permet aussi à Utilisateur de réaliser divers actions et d'afficher les écrans (EcranPrincipal et Popup).
    \end{itemize}

    \item L'objet \textit{dealer} stocke et fournit à \textit{gui} les informations nécessaires à l'affichage des écrans. Il peut ajouter de nouveaux objets ou trames ou supprimer des éléments sélectionnés par Utilisateur et stockés dans \textit{basket}.

    \item L'objet \textit{object} permet de stocker et récupérer les informations d'une instance d'un objet, tout comme l'objet \textit{frame} permettant de stocker et récupérer les informations d'une instance d'une trame.

    \item L'objet \textit{basket} contient l'ensemble des objets et des trames sélectionnés par Utilisateur.

    \item L'objet \textit{sender} permet d'envoyer les trames sélectionnées par Utilisateur et stockées dans \textit{basket}.

    \item L'objet \textit{sniffer} permet de récupérer les trames reçues par le bus CAN et de les stocker dans l'objet \textit{logger}. 

    \item L'objet \textit{logger} permet de stocker les trames reçues par le bus CAN et de notifier \textit{gui} qu'une nouvelle trame doit être affichée.

    \item L'objet \textit{network} permet d'informer \textit{gui} de l'état de connexion entre l'application {\nomApplication} et le programme {\nomLogiciel}.
\end{itemize}

\subsection{Diagrammes de séquence}

\subsubsection{\emph{CU Échanger des trames CAN}}
\begin{figure}[H]
	\centering
	\includegraphics[max width=\textwidth,max height=.9\textheight]{../animUML/Passerelle_CAN-sequence-CU_Échanger_des_trames_CAN}
	\caption{Diagramme de séquence du \emph{CU Échanger des trames CAN}}
	\label{fig:inter-CU_Échanger_des_trames_CAN}
\end{figure}
Le diagramme de la \autoref{fig:inter-CU_Échanger_des_trames_CAN} représente le diagramme de séquence du \emph{CU Échanger des trames CAN}.
\\
Utilisateur démarre le SàE, et il commence à recevoir continuellement des trames. Utilisateur peut ensuite, à partir de l'application {\nomApplication}, exercer différentes actions : ajouter un objet, enregistrer une trame, envoyer des trames, arrêter l'envoi de trames, supprimer un objet et/ou une trame et stopper le SàE. \\ 
Sur le diagramme, Utilisateur effectue toutes ces actions dans un ordre précis mais en réalité, il peut effectuer ces actions dans l'ordre qu'il souhaite, ou bien, n'en effectuer aucune. \\
En cas de perte de connexion entre l'application {\nomApplication} et le programme {\nomLogiciel}, Utilisateur peut les reconnecter ensemble. Ceci ne figure cependant pas sur le diagramme de la \autoref{fig:inter-CU_Échanger_des_trames_CAN}, qui ne montre que le scénario nominal.\\

\subsubsection{\emph{CU Démarrer le SàE - Scénario nominal}}
\begin{figure}[H]
	\centering
	\includegraphics[max width=\textwidth,max height=.9\textheight]{../animUML/Passerelle_CAN-sequence-CU_Démarrer_le_SàE_-_Scénario_nominal}
	\caption{Diagramme de séquence du \emph{CU Démarrer le SàE - Scénario nominal}}
	\label{fig:inter-CU_Démarrer_le_SàE_-_Scénario_nominal}
\end{figure}
Le diagramme de la \autoref{fig:inter-CU_Démarrer_le_SàE_-_Scénario_nominal} représente le diagramme de séquence du \emph{CU Démarrer le SàE - Scénario nominal}.
\\
Pour rappel, en dehors de la portée du système, Utilisateur met en fonctionnement Tableau de Bord, connecte la Raspberry PI au bus CAN et met la Raspberry PI sous tension.\\
Utilisateur doit ensuite démarrer manuellement le programme {\nomLogiciel} via une connexion SSH avec la Raspberry PI. Quand le programme {\nomLogiciel} est démarré, la LED de la Raspberry PI renvoie l'information à Utilisateur. Le programme {\nomLogiciel} commence alors à lire les trames du bus CAN.\\
Ensuite, Utilisateur démarre l'application {\nomApplication} et EcranPrincipal s'affiche. L'application {\nomApplication} charge l'ensemble des éléments créés lors des précédentes utilisations et les affiche sur EcranPrincipal.\\
Enfin, l'application {\nomApplication} se connecte au programme {\nomLogiciel} et EcranPrincipal se met à jour.\\
Il est possible que la connexion entre l'application {\nomApplication} et le programme {\nomLogiciel} échoue. Dans ce cas, PopupReconnexion s'affiche et demande à Utilisateur s'il souhaite réessayer. Ce scénario n'est pas représenter sur la \autoref{fig:inter-CU_Démarrer_le_SàE_-_Scénario_nominal}.\\

\subsubsection{\emph{CU Reconnecter application CANdroid - Scénario nominal}}
\begin{figure}[H]
	\centering
	\includegraphics[max width=\textwidth,max height=.9\textheight]{../animUML/Passerelle_CAN-sequence-CU_Reconnecter_application_CANdroid_-_Scénario_nominal}
	\caption{Diagramme de séquence du \emph{CU Reconnecter application CANdroid - Scénario nominal}}
	\label{fig:inter-CU_Reconnecter_application_CANdroid_-_Scénario_nominal}
\end{figure}
Le diagramme de la \autoref{fig:inter-CU_Reconnecter_application_CANdroid_-_Scénario_nominal} représente le diagramme de séquence du \emph{CU Reconnecter application CANdroid - Scénario nominal}.
\\
Si la connexion entre l'application {\nomApplication} et le programme {\nomLogiciel} échoue, Utilisateur peut demander de les reconnecter. Dans ce cas, l'application {\nomApplication} affiche PopupDemandeReconnexion. PopupDemandeReconnexion s'affiche également automatiquement en cas de perte de connexion imprévue.
\\
Si Utilisateur confirme la demande de reconnexion, EcranPrincipal s'affiche et l'application {\nomApplication} tente alors de se reconnecter au programme {\nomLogiciel}. Si la connexion aboutit, l'application {\nomApplication} met à jour EcranPrincipal. \\
Dans le cas général, Utilisateur peut également refuser la reconnexion, et l'application {\nomApplication} affiche simplement EcranPrincipal sans tenter de se reconnecter. Utilisateur peut aussi redemander la reconnexion suite à un échec. Ces cas de figure ne sont pas présentés sur ce scénario, mais sont explicitement décrits dans le CU "Reconnecter l'application {\nomApplication}" du dossier de spécification [\hyperref[SPEC]{dossier\_de\_specification\_SPEC\_B1\_2024}].

\subsubsection{\emph{CU Recevoir des trames - Scénario nominal}}
\begin{figure}[H]
	\centering
	\includegraphics[max width=\textwidth,max height=.9\textheight]{../animUML/Passerelle_CAN-sequence-CU_Recevoir_des_trames_-_Scénario_nominal}
	\caption{Diagramme de séquence du \emph{CU Recevoir des trames - Scénario nominal}}
	\label{fig:inter-CU_Recevoir_des_trames_-_Scénario_nominal}
\end{figure}
Le diagramme de la \autoref{fig:inter-CU_Recevoir_des_trames_-_Scénario_nominal} représente le diagramme de séquence du \emph{CU Recevoir des trames - Scénario nominal}.
\\
La première étape du scénario nominal du CU "Recevoir des trames" du dossier de spécification [\hyperref[SPEC]{dossier\_de\_specification\_SPEC\_B1\_2024}] n'est pas représentée dans le diagramme de séquence car elle est implicite. On suppose que Tableau de Bord fonctionne et est capable d'émettre des trames sur le bus CAN. \\
Périodiquement, \textit{sniffer} reçoit une trame de Tableau de Bord. Lorsqu'il reçoit une trame, il transmet celle-ci à \textit{logger} qui informe ensuite \textit{gui} de la réception d'une nouvelle trame. Cela permet ainsi à \textit{gui} de récupérer et afficher cette trame sur EcranPrincipal. \\

Le diagramme de la \autoref{fig:inter-CU_Recevoir_des_trames_-_Scénario_nominal} représente la réception d'une trame, cependant, il est possible que \textit{sniffer} reçoive beaucoup de trames en peu de temps. Dans ce cas, l'opération "notifyNewFrames()" est appelée périodiquement et \textit{gui} récupère toutes les trames reçues depuis le dernier appel de "getFrames()". Ainsi, les opérations "notifyNewFrames()" et "getFrames()" sont au pluriel car il est possible de récupérer un paquet de trames. 
Pour finir, comme on peut le voir dans la \autoref{fig:inter-CU_Interagir_avec_le_sniffer_-_Scénario_nominal}, Utilisateur a la possibilité de mettre en pause la réception des trames, ce qui suspend temporairement la boucle. \\

\subsubsection{\emph{CU Interagir avec le sniffer - Scénario nominal}}
\begin{figure}[H]
	\centering
	\includegraphics[max width=\textwidth,max height=.9\textheight]{../animUML/Passerelle_CAN-sequence-CU_Interagir_avec_le_sniffer_-_Scénario_nominal}
	\caption{Diagramme de séquence du \emph{CU Interagir avec le sniffer - Scénario nominal}}
	\label{fig:inter-CU_Interagir_avec_le_sniffer_-_Scénario_nominal}
\end{figure}
Le diagramme de la \autoref{fig:inter-CU_Interagir_avec_le_sniffer_-_Scénario_nominal} représente le diagramme de séquence du \emph{CU Interagir avec le sniffer - Scénario nominal}.
\\
Utilisateur peut interagir avec le sniffer et réaliser plusieurs actions : 
\begin{itemize}
    \item Revenir en haut du sniffer pour voir apparaître les nouvelles trames. 
    \item Nettoyer le sniffer c'est-à-dire supprimer toutes les trames présentes dans le sniffer. 
    \item Lancer ou mettre en pause la réception des trames. 
    \item Exporter les trames dans un fichier log en dehors de l'application {\nomApplication}. Cette action est uniquement possible si Utilisateur a mis en pause la réception des trames.
\end{itemize}
De plus, la \autoref{fig:inter-CU_Interagir_avec_le_sniffer_-_Scénario_nominal} indique un ordre spécifique pour les actions à effectuer, à savoir l'ordre du CU "Interagir avec le sniffer" du dossier de spécification [\hyperref[SPEC]{dossier\_de\_specification\_SPEC\_B1\_2024}]. Cependant, dans la version comprenant les variantes, il est indiqué que les actions peuvent être effectuées dans l'ordre souhaité, à l'exception de l'exportation des trames, qui doit être réalisée après la mise en pause du sniffer. En d'autres termes, Utilisateur a la liberté d'effectuer les actions dans l'ordre qui lui convient, excepté pour l'exportation des trames qui nécessite une mise en pause du sniffer au préalable.

\subsubsection{\emph{CU Ajouter un objet - Scénario nominal}}
\begin{figure}[H]
	\centering
	\includegraphics[max width=\textwidth,max height=.9\textheight]{../animUML/Passerelle_CAN-sequence-CU_Ajouter_un_objet_-_Scénario_nominal}
	\caption{Diagramme de séquence du \emph{CU Ajouter un objet - Scénario nominal}}
	\label{fig:inter-CU_Ajouter_un_objet_-_Scénario_nominal}
\end{figure}
Le diagramme de la \autoref{fig:inter-CU_Ajouter_un_objet_-_Scénario_nominal} représente le diagramme de séquence du \emph{CU Ajouter un objet - Scénario nominal}.
\\
Utilisateur demande l'ajout d'un nouvel objet sur l'application {\nomApplication}. Si le nombre d'objets existants dans l'application {\nomApplication} est inférieur à la limite maximale, PopupAjoutObjet s'affiche. Ce pop-up permet à Utilisateur de donner un nom à l'objet qu'il souhaite ajouter. Une fois le nom saisi, Utilisateur valide la demande d'ajout. Enfin, l'objet est ajouté aux objets déjà présents sur l'application {\nomApplication} et EcranPrincipal est mis à jour en conséquence. \\
Dans le cas général, Utilisateur peut choisir de laisser le nom d'objet par défaut présent dans <champNomObjet> pré-rempli. Ce nom est obtenu grâce à l'appel de l'opération "getMaxIdObject()" qui permet de donner le numéro du nom de l'objet. Par exemple, si l'opération renvoie le nombre "4" alors le Nom d'objet par défaut est "Objet\_4".\\
De plus, si Utilisateur a entré un nom déjà utilisé par un autre objet, l'application {\nomApplication} affiche PopupErreurAjoutObjet. S'il essaie d'ajouter un objet alors que le nombre d'objet maximum est atteint, l'application {\nomApplication} affiche PopupErreurNombreObjet. Ces cas de figure ne sont pas présentés sur ce scénario, mais sont explicitement décrits dans le CU "Ajouter un objet" du dossier de spécification [\hyperref[SPEC]{dossier\_de\_specification\_SPEC\_B1\_2024}].


\subsubsection{\emph{CU Ajouter une trame - Scénario nominal}}
\begin{figure}[H]
	\centering
	\includegraphics[max width=\textwidth,max height=.9\textheight]{../animUML/Passerelle_CAN-sequence-CU_Ajouter_une_trame_-_Scénario_nominal}
	\caption{Diagramme de séquence du \emph{CU Ajouter une trame - Scénario nominal}}
	\label{fig:inter-CU_Ajouter_une_trame_-_Scénario_nominal}
\end{figure}
Le diagramme de la \autoref{fig:inter-CU_Ajouter_une_trame_-_Scénario_nominal} représente le diagramme de séquence du \emph{CU Ajouter une trame - Scénario nominal}.
\\
Lorsque Utilisateur ouvre le menu de l'objet, EcranPrincipal se met à jour. Utilisateur entre une nouvelle trame dans le menu de l'objet associé et l'enregistre si le nombre de trames dans cet objet est inférieur à la limite maximale autorisée, pour l'enregistrement des trames. L'application {\nomApplication} affiche PopupModeEnvoiTrame, permettant à Utilisateur de définir le Mode Envoi de la trame. Utilisateur définit et valide son choix, la trame s'enregistre et EcranPrincipal s'affiche.\\

Dans le cas général, la saisie de la trame peut être incorrecte, l'application {\nomApplication} affiche donc PopupErreurSaisieTrame, qui indique le bon format des trames à enregistrer. Utilisateur peut également décider d'ajouter une trame alors que le nombre de trame maximum est atteint, dans ce cas, l'application {\nomApplication} afficher PopupErreurNombreTrame. De plus, dans la \autoref{fig:inter-CU_Ajouter_une_trame_-_Scénario_nominal}, le Mode Envoi est défini par défaut en mode cyclique avec une périodicité par défaut de 100 ms et Utilisateur le définit en mode ponctuel. Cependant, Utilisateur peut choisir de ne pas effectuer ce changement. Ces cas de figure ne sont pas présentés sur ce scénario, mais sont explicitement décrits dans le CU "Enregistrer une trame" du dossier de spécification [\hyperref[SPEC]{dossier\_de\_specification\_SPEC\_B1\_2024}].

\subsubsection{\emph{CU Envoyer des trames - Scénario nominal}}
\begin{figure}[H]
	\centering
	\includegraphics[max width=\textwidth,max height=.9\textheight]{../animUML/Passerelle_CAN-sequence-CU_Envoyer_des_trames_-_Scénario_nominal}
	\caption{Diagramme de séquence du \emph{CU Envoyer des trames - Scénario nominal}}
	\label{fig:inter-CU_Envoyer_des_trames_-_Scénario_nominal}
\end{figure}
Le diagramme de la \autoref{fig:inter-CU_Envoyer_des_trames_-_Scénario_nominal} représente le diagramme de séquence du \emph{CU Envoyer des trames - Scénario nominal}.
\\
Utilisateur ouvre le menu de l'objet et sélectionne la trame qu'il souhaite envoyer. L'application {\nomApplication} affiche ensuite EcranPrincipal. Utilisateur commence l'envoi de la trame sélectionnée.  Pour que l'envoi commence, \textit{sender} a besoin des informations concernant la trame choisie. Ces informations lui sont transmises grâce à \textit{gui} qui les récupère de \textit{dealer}. Dès que l'application passe en Mode Envoi, EcranPrincipal est mis à jour en conséquence. Afin d'afficher la trame émise dans le sniffer de l'application, il est nécessaire de transmettre la trame à \textit{logger}. Le diagramme de la \autoref{fig:inter-CU_Envoyer_des_trames_-_Scénario_nominal} ne représente que l'envoi d'une trame mais il est possible d'envoyer plusieurs trames à la fois. \\
En réalité, les opérations "askSendingState()" et "setSendingState(sendingState : SendingState)" sont appelées périodiquements pour informer \textit{gui} de l'état de l'application {\nomApplication}. \\

Dans le cas général, si Utilisateur souhaite désélectionner un élément, il lui suffit de cliquer à nouveau sur ce dernier. Si un objet a été sélectionné, il n'est pas possible de désélectionner individuellement une trame qui fait partie de ce même objet. \\
Pour chaque trame sélectionnée, \textit{gui} récupère les informations de la trame grâce à l'opération "getFrames()" de \textit{dealer}. \\
De plus, si les trames sélectionnées ont des modes d'envoi périodiques, alors l'appel des opérations "send()", "setFrame(frame)", "notifyNewFrames()", "getFrames()" et "displayMainScreen()" se répète jusqu'à la demande d'arrêt. \\
Ces cas de figure ne sont pas présentés sur ce scénario, mais sont explicitement décrits dans le CU "Envoyer des trames" du dossier de spécification [\hyperref[SPEC]{dossier\_de\_specification\_SPEC\_B1\_2024}].

\subsubsection{\emph{CU Arrêter envoi des trames - Scénario nominal}}
\begin{figure}[H]
	\centering
	\includegraphics[max width=\textwidth,max height=.9\textheight]{../animUML/Passerelle_CAN-sequence-CU_Arrêter_envoi_des_trames_-_Scénario_nominal}
	\caption{Diagramme de séquence du \emph{CU Arrêter envoi des trames - Scénario nominal}}
	\label{fig:inter-CU_Arrêter_envoi_des_trames_-_Scénario_nominal}
\end{figure}
Le diagramme de la \autoref{fig:inter-CU_Arrêter_envoi_des_trames_-_Scénario_nominal} représente le diagramme de séquence du \emph{CU Arrêter envoi des trames - Scénario nominal}.
\\
Lorsque Utilisateur demande l'arrêt d'envoi des trames, l'application {\nomApplication} affiche PopupArretEnvoi. Utilisateur valide la demande d'arrêt, l'application {\nomApplication} stoppe l'envoi de trames et met à jour EcranPrincipal en conséquence.\\
Dans le cas général, si Utilisateur refuse la demande d'arrêt d'envoi des trames, l'application {\nomApplication} ne s'arrête pas et continue à envoyer des trames. Ce cas de figure est explicitement décrit dans le CU "Arrêter l'envoi des trames" du dossier de spécification [\hyperref[SPEC]{dossier\_de\_specification\_SPEC\_B1\_2024}].

\subsubsection{\emph{CU Supprimer un élément - Scénario nominal}}
\begin{figure}[H]
	\centering
	\includegraphics[max width=\textwidth,max height=.9\textheight]{../animUML/Passerelle_CAN-sequence-CU_Supprimer_un_élément_-_Scénario_nominal}
	\caption{Diagramme de séquence du \emph{CU Supprimer un élément - Scénario nominal}}
	\label{fig:inter-CU_Supprimer_un_élément_-_Scénario_nominal}
\end{figure}
Le diagramme de la \autoref{fig:inter-CU_Supprimer_un_élément_-_Scénario_nominal} représente le diagramme de séquence du \emph{CU Supprimer un élément - Scénario nominal}.
\\
Utilisateur sélectionne un objet et toutes les trames qui lui sont associées sont automatiquement sélectionnées en conséquence. Par la suite, L'application {\nomApplication} se met à jour. Après avoir sélectionné les éléments à supprimer, l'application {\nomApplication} affiche PopupSuppressionElement et Utilisateur valide la suppression. Une fois la suppression confirmée, l'application {\nomApplication} met à jour EcranPrincipal pour refléter les changements.\\

Dans le cas général, Utilisateur a la possibilité de sélectionner une trame sans nécessairement sélectionner l'objet qui la contient, et de la supprimer. De plus, il est possible de sélectionner plusieurs éléments simultanément pour les supprimer, et l'application {\nomApplication} met à jour EcranPrincipal en conséquence. Ces cas de figure ne sont pas présentés sur ce scénario, mais sont explicitement décrits dans le CU "Supprimer un élément" du dossier de spécification [\hyperref[SPEC]{dossier\_de\_specification\_SPEC\_B1\_2024}]. 

\subsubsection{\emph{CU Stopper le SàE - Scénario nominal}}
\begin{figure}[H]
	\centering
	\includegraphics[max width=\textwidth,max height=.9\textheight]{../animUML/Passerelle_CAN-sequence-CU_Stopper_le_SàE_-_Scénario_nominal}
	\caption{Diagramme de séquence du \emph{CU Stopper le SàE - Scénario nominal}}
	\label{fig:inter-CU_Stopper_le_SàE_-_Scénario_nominal}
\end{figure}
Le diagramme de la \autoref{fig:inter-CU_Stopper_le_SàE_-_Scénario_nominal} représente le diagramme de séquence du \emph{CU Stopper le SàE - Scénario nominal}.
\\ 
Pour arrêter le SàE, Utilisateur commence par arrêter l'application {\nomApplication} grâce à l'opération "stopCANdroid()". Il arrête ensuite le programme {\nomLogiciel} avec l'opération "stopCANgateway()". Cette opération se charge d'appeler l'opération "stopReading()" qui arrête la lecture en continue des trames CAN. Pour finir, le programme {\nomLogiciel} informe Utilisateur que le SàE est arrêté grâce à l'opération "signalState(programmeState : ProgrammeState). \\

Dans le cas général, Utilisateur peut décider d'utiliser seulement le Smartphone avec l'application {\nomApplication}, il n'a donc pas besoin d'arrêter le programme {\nomLogiciel}. \\


\subsection{Types de données}

\begin{figure}[H]
	\centering
	\includegraphics[max width=\textwidth,max height=.9\textheight]{../animUML/Passerelle_CAN-datatypes}
	\caption{Diagramme des types de données}
	\label{fig:datatypes}
\end{figure}
Le diagramme de la \autoref{fig:datatypes} représente les types de données utilisés.
\newline
\subsubsection{Description de l'énumération  IdScreenPopUp}

Cette énumération représente les différents types de pop-ups utilisés dans le système. Chaque valeur de l'énumération correspond à un type de pop-up spécifique, tel que, par exemple, l'ajout d'objet, l'arrêt d'envoi ou la demande de reconnexion.
\newline

En voici le détail des énumérations:
\enumIdScreenPopUpLiteralDescriptions

\subsubsection{Description de l'énumération ProgramState}

Cette énumération représente l'état du programme, qui peut être soit "OFF" lorsque le programme est arrêté, soit "ON" lorsque le programme est en cours d'exécution.
\newline

En voici le détail des énumérations:
\enumProgramStateLiteralDescriptions

\subsubsection{Description de l'énumération NetworkState}

Cette énumération représente l'état de la connexion réseau, qui peut être soit "CONNECTED" lorsque la connexion est établie, soit "NOT\_CONNECTED" lorsque la connexion est perdue.
\newline

En voici le détail des énumérations:
\enumNetworkStateLiteralDescriptions

\subsubsection{Description de l'énumération SenderMode}

Cette énumération représente le Mode Envoi des données, qui peut être soit "PUNCTUAL" pour un envoi ponctuel, soit "CYCLIC" pour un envoi cyclique.
\newline

En voici le détail des énumérations:
\enumSenderModeLiteralDescriptions

\subsubsection{Description de l'énumération SendingState}

Cette énumération représente l'état d'envoi des données, qui peut être soit "ON" lorsque l'envoi est activé, soit "OFF" lorsque l'envoi est désactivé.
\newline

En voici le détail des énumérations:
\enumSendingStateLiteralDescriptions

\subsubsection{Descriptions des autres types de données}

Voici une liste des autres types de données utilisés dans le dossier ainsi que leur description :
\begin{itemize}
    \item \textbf{bool} : booléen, peut prendre la valeur true ou false.
    \item \textbf{CANFrame} : structure de données représentant un trame CAN. Elle contient les champs suivants :
    \begin{itemize}
        \item \textbf{id} : identifiant de la trame. C'est un entier non signé de 32 bits.
        \item \textbf{size} : taille de la trame. C'est un entier non signé de 8 bits.
        \item \textbf{data} : tableau de 8 octets non signés contenant les données de la trame.
    \end{itemize}
    \item \textbf{IdElement} : identifiant des objets et des trames. C'est un entier non signé de 16 bits. L'ID d'un objet est de la forme XX000 où XX représente l'ID de l'objet. L'ID d'une trame est de la forme XXYYY où XX représente l'ID de l'objet et YYY représente l'ID de la trame. Par exemple, si l'objet "Clignotant droit" a pour ID 06000, alors la première trame associée à cet objet aura pour ID 06001.
    \item \textbf{String} : chaîne de caractères.
    \item \textbf{StructSenderMode} : structure de données contenant le Mode Envoi (enumération SenderMode) et la périodicité si le Mode Envoi est cyclique.
\end{itemize}

\subsection{Classes}

\subsubsection{Vue générale}

\begin{figure}[H]
	\centering
	\includegraphics[max width=\textwidth,max height=.9\textheight]{../animUML/Passerelle_CAN-classes}
	\caption{Diagramme de classes}
	\label{fig:classes}
\end{figure}
Le diagramme de la \autoref{fig:classes} représente les classes du système.

\subsubsection{La classe Utilisateur}

Le diagramme de la \autoref{fig:class-Utilisateur} représente la classe Utilisateur.
\begin{figure}[H]
	\centering
	\includegraphics[max width=\textwidth,max height=.9\textheight]{../animUML/Passerelle_CAN-class-Utilisateur}
	\caption{Diagramme de la classe Utilisateur}
	\label{fig:class-Utilisateur}
\end{figure}
\input{sections/2_Conception_generale/Description_classes/Utilisateur.tex}

\paragraph{Attributs}
\classUtilisateurProperties
\paragraph{Services offerts}
\classUtilisateurOperations
\subsubsection{La classe GUI}

Le diagramme de la \autoref{fig:class-GUI} représente la classe GUI.
\begin{figure}[H]
	\centering
	\includegraphics[max width=\textwidth,max height=.9\textheight]{../animUML/Passerelle_CAN-class-GUI}
	\caption{Diagramme de la classe GUI}
	\label{fig:class-GUI}
\end{figure}
% ----------------------------
% Description de la classe GUI
% ----------------------------
% TODO :
\paragraph{Philosophie de conception}
La classe GUI permet de gérer les interfaces utilisateur de l'application {\nomApplication}.\\

\paragraph{Attributs}
\classGUIProperties
\paragraph{Services offerts}
\classGUIOperations
\paragraph{Description comportementale}
\begin{figure}[H]
	\centering
	\includegraphics[max width=\textwidth,max height=.9\textheight]{../animUML/Passerelle_CAN-gui-SM}
	\caption{Machine à états de \emph{GUI}}
	\label{fig:sm-gui}
\end{figure}
Le diagramme de la \autoref{fig:sm-gui} représente la machine à états de \emph{GUI}.
\newline
Lorsque l'application {\nomApplication} démarre, l'état par défaut de la MAE est l'état MainScreen. Entrer dans cet état appelle systématiquement l'opération pour afficher EcranPrincipal. Un certain nombre d'opérations peuvent être appelées dans cet état. Elles sont internes à l'état. \newline

L'application {\nomApplication} entre dans l'état Popup lorsque l'opération associée à l'apparition d'un pop-up est appelée. Entrer dans cet état appelle l'opération pour afficher le pop-up correspondant. L'appel de l'opération "rejectRequest()" permet de ramener l'application {\nomApplication} à l'état MainScreen. L'appel de l'opération "acceptRequest()" permet de ramener l'application {\nomApplication} à l'état MainScreen et de lancer l'opération correspondante à l'action de Utilisateur selon le pop-up qui était précédemment affiché sur Smartphone. 

\subsubsection{La classe UI}

Le diagramme de la \autoref{fig:class-UI} représente la classe UI.
\begin{figure}[H]
	\centering
	\includegraphics[max width=\textwidth,max height=.9\textheight]{../animUML/Passerelle_CAN-class-UI}
	\caption{Diagramme de la classe UI}
	\label{fig:class-UI}
\end{figure}
% ---------------------------
% Description de la classe UI
% ---------------------------
% TODO :
 \paragraph{Philosophie de conception}
La classe UI permet de gérer les interactions de Utilisateur avec le programme {\nomLogiciel}.\\

\paragraph{Attributs}
\classUIProperties
\paragraph{Services offerts}
\classUIOperations
\subsubsection{La classe Dealer}

Le diagramme de la \autoref{fig:class-Dealer} représente la classe Dealer.
\begin{figure}[H]
	\centering
	\includegraphics[max width=\textwidth,max height=.9\textheight]{../animUML/Passerelle_CAN-class-Dealer}
	\caption{Diagramme de la classe Dealer}
	\label{fig:class-Dealer}
\end{figure}
% -------------------------------
% Description de la classe Dealer
% -------------------------------
% TODO :
 \paragraph{Philosophie de conception}
La classe Dealer permet de gérer les objets et trames enregistrés sur l'application {\nomApplication}.\\

\paragraph{Attributs}
\classDealerProperties
\paragraph{Services offerts}
\classDealerOperations
\subsubsection{La classe Logger}

Le diagramme de la \autoref{fig:class-Logger} représente la classe Logger.
\begin{figure}[H]
	\centering
	\includegraphics[max width=\textwidth,max height=.9\textheight]{../animUML/Passerelle_CAN-class-Logger}
	\caption{Diagramme de la classe Logger}
	\label{fig:class-Logger}
\end{figure}
% -------------------------------
% Description de la classe Logger
% -------------------------------
% TODO :
 \paragraph{Philosophie de conception}
La classe Logger permet de mémoriser les trames reçues par Sniffer et de les enregistrer dans un fichier.\\

\paragraph{Attributs}
\classLoggerProperties
\paragraph{Services offerts}
\classLoggerOperations
\subsubsection{La classe Object}

Le diagramme de la \autoref{fig:class-Object} représente la classe Object.
\begin{figure}[H]
	\centering
	\includegraphics[max width=\textwidth,max height=.9\textheight]{../animUML/Passerelle_CAN-class-Object}
	\caption{Diagramme de la classe Object}
	\label{fig:class-Object}
\end{figure}
% -------------------------------
% Description de la classe Object
% -------------------------------
% TODO :
 \paragraph{Philosophie de conception}
La classe Object représente une instance d'un objet \textit{object}. Il est caractérisé par son nom, son ID et les trames qui lui sont associées. Par cette classe, on peut récupérer les trames qui lui sont associées, en ajouter de nouvelles ou en supprimer.\\

\paragraph{Attributs}
\classObjectProperties
\paragraph{Services offerts}
\classObjectOperations
\subsubsection{La classe Frame}

Le diagramme de la \autoref{fig:class-Frame} représente la classe Frame.
\begin{figure}[H]
	\centering
	\includegraphics[max width=\textwidth,max height=.9\textheight]{../animUML/Passerelle_CAN-class-Frame}
	\caption{Diagramme de la classe Frame}
	\label{fig:class-Frame}
\end{figure}
% ------------------------------
% Description de la classe Frame
% ------------------------------
% TODO :
 \paragraph{Philosophie de conception}
La classe Frame représente une instance de l'objet \textit{frame}. Elle est caractérisée par sa trame (de type CANFrame) et son Mode Envoi. Avec cette classe, on peut récupérer le Mode Envoi de la trame mais aussi le modifier.\\

\paragraph{Attributs}
\classFrameProperties
\paragraph{Services offerts}
\classFrameOperations
\subsubsection{La classe Sniffer}

Le diagramme de la \autoref{fig:class-Sniffer} représente la classe Sniffer.
\begin{figure}[H]
	\centering
	\includegraphics[max width=\textwidth,max height=.9\textheight]{../animUML/Passerelle_CAN-class-Sniffer}
	\caption{Diagramme de la classe Sniffer}
	\label{fig:class-Sniffer}
\end{figure}
% --------------------------------
% Description de la classe Sniffer
% --------------------------------
% TODO :
 \paragraph{Philosophie de conception}
La classe Sniffer gère la lecture des trames sur le bus CAN et l'envoi des trames lues vers l'application {\nomApplication}.\\

\paragraph{Attributs}
\classSnifferProperties
\paragraph{Services offerts}
\classSnifferOperations
\paragraph{Description comportementale}
\begin{figure}[H]
	\centering
	\includegraphics[max width=\textwidth,max height=.9\textheight]{../animUML/Passerelle_CAN-sniffer-SM}
	\caption{Machine à états de \emph{Sniffer}}
	\label{fig:sm-sniffer}
\end{figure}
Le diagramme de la \autoref{fig:sm-sniffer} représente la machine à états de \emph{Sniffer}.
\newline
Lors du lancement de l'application {\nomApplication}, la machine à états est dans l'état IDLE. Dès que l'ensemble des connexions entre le Smartphone, la Raspberry PI et Tableau de Bord sont faites, la machine à états passe à l'état Listening. Dans cet état, Sniffer reçoit les trames en continu, comme le montre la fonction "receiveFrame()". \newline
Lorsque Utilisateur souhaite mettre en pause l'arrivée des trames, la machine repasse dans l'état IDLE via l'opération "stopListening()". Il peut ainsi retourner en écoute lors de la reprise du sniffer via l'opération "startListening()". \newline 
\subsubsection{La classe Network}

Le diagramme de la \autoref{fig:class-Network} représente la classe Network.
\begin{figure}[H]
	\centering
	\includegraphics[max width=\textwidth,max height=.9\textheight]{../animUML/Passerelle_CAN-class-Network}
	\caption{Diagramme de la classe Network}
	\label{fig:class-Network}
\end{figure}
% --------------------------------
% Description de la classe Network
% --------------------------------
% TODO :
 \paragraph{Philosophie de conception}
La classe Network permet d'interagir avec GUI et de déterminer si la connexion entre le smartphone et la Rapsberry Pi est établie. \\

\paragraph{Attributs}
\classNetworkProperties
\paragraph{Services offerts}
\classNetworkOperations
\subsubsection{La classe Basket}

Le diagramme de la \autoref{fig:class-Basket} représente la classe Basket.
\begin{figure}[H]
	\centering
	\includegraphics[max width=\textwidth,max height=.9\textheight]{../animUML/Passerelle_CAN-class-Basket}
	\caption{Diagramme de la classe Basket}
	\label{fig:class-Basket}
\end{figure}
% -------------------------------
% Description de la classe Basket
% -------------------------------
% TODO :
\paragraph{Philosophie de conception}
La classe Basket permet d'enregistrer les éléments sélectionnés par Utilisateur.\\

\paragraph{Attributs}
\classBasketProperties
\paragraph{Services offerts}
\classBasketOperations
\subsubsection{La classe Sender}

Le diagramme de la \autoref{fig:class-Sender} représente la classe Sender.
\begin{figure}[H]
	\centering
	\includegraphics[max width=\textwidth,max height=.9\textheight]{../animUML/Passerelle_CAN-class-Sender}
	\caption{Diagramme de la classe Sender}
	\label{fig:class-Sender}
\end{figure}
% -------------------------------
% Description de la classe Sender
% -------------------------------
% TODO :
 \paragraph{Philosophie de conception}
La classe Sender permet d'envoyer les trames sur le bus CAN.\\

\paragraph{Attributs}
\classSenderProperties
\paragraph{Services offerts}
\classSenderOperations
\paragraph{Description comportementale}
\begin{figure}[H]
	\centering
	\includegraphics[max width=\textwidth,max height=.9\textheight]{../animUML/Passerelle_CAN-sender-SM}
	\caption{Machine à états de \emph{Sender}}
	\label{fig:sm-sender}
\end{figure}
Le diagramme de la \autoref{fig:sm-sender} représente la machine à états de \emph{Sender}.
\newline
Lorsque l'application {\nomApplication} est en fonctionnement, la machine à états est dans l'état IDLE. L'opération "setSendingState()" renvoi la valeur de l'énumération correspondant à l'état d'envoi en cours. Ici, les trames ne sont pas envoyées sur Tableau de Bord, donc le retour est SENDING\_STATE\_OFF. \newline
Dès lors que Utilisateur souhaite envoyer des trames via l'opération "startSending()", la machine à états passe dans l'état Sending. État durant lequel, l'opération "setSendingState()" va retourner SENDING\_STATE\_ON afin d'informer que les trames sont en cours d'envoi, comme il est possible de le voir avec "do/send()" qui agit constamment. L'état est quitté lorsque l'opération "stopSending()" est appelée, cela arrête l'envoi des trames et remet l'état de la machine à états sur IDLE.\newline 
