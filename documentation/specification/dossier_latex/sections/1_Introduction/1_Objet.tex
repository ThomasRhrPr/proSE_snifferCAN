% -----------
% OBJET {p.TODO}
% -----------
% TODO : texte à revoir 
\subsection{Objet} %1.1
Ce dossier de spécifications a pour objectif de définir les fonctionnalités et exigences attendues par la société {\client} (désignée dans la suite du document comme le \hyperref[client_kereval]{Client}) pour le développement logiciel du prototype {\guillemotleft} Passerelle Android-CAN vers banc CAN réel ou simulé {\guillemotright}. 
Dans le contexte de développement d'un prototype, ce dossier de spécifications se focalise sur une étape du cycle de vie du produit, à savoir l'envoi et la réception de trames depuis une application Android, {\nomApplication}, vers un Simulateur logiciel (\hyperref[ICSim]{ICSim}) ou physique (Banc de test), en n'adressant que les fonctionnalités principales attendues pour ce prototype.
\medskip 

Ce document permettra aux équipes de conception, de réalisation et de test de la société {\client} de prendre connaissance de manière précise et détaillée les différentes parties informatiques du prototype.
Les fonctionnalités et exigences présentées dans ce document ont été déterminées suite à l'étude du cahier des charges défini avec le Client et des rencontres avec le représentant du Client, A. Ribault (voir [\hyperref[cdc_kereval]{CdC\_KEREVAL\_2023}]). 
\medskip 

Ce dossier de spécifications s'inspire de la norme [\hyperref[norme]{ISO/IEC/IEEE 29148 : 2018}]. Il utilise des schémas et figures respectant la norme [\hyperref[uml_version]{UML\_2.5}]. 
De même, il respecte les exigences du Plan d'Assurance Qualité Logicielle (PAQL) défini par la société {\teamName} [\hyperref[paql]{PAQL\_B1\_2024}].
\medskip