% ---------------
% REFERENCE {p.5}
% ---------------
\subsection{Références} %1.4
Voici un tableau récapitulatif des documents utilisés pour le dossier de spécifications ainsi que les liens permettant d'accéder aux fichiers.
\medskip
% TODO : ajouter toutes les références nécessaires
\begin{longtable}[l]{|m{0.3\linewidth}|m{0.6\linewidth}|} 
    \hline
        \centering [CdC\_KEREVAL\_2023] \label{cdc_kereval} & Société {\client} {\guillemotleft} Cahier des charges : développement d'une Passerelle Android-CAN vers banc CAN réel ou simulé {\guillemotright}, 2023. \\
    \hline
        \centering [ISO/IEC/IEEE 29148 : 2018] \label{norme} & International standard, systems and software engineering life cycle processes requirements engineering, 2018, \href{https://standards.ieee.org/standard/29148-2018.html}{https://standards.ieee.org/standard/29148-2018.html}.\\
    \hline
        \centering [UML\_2.5] \label{uml_version} & OMG, Unified Modeling Language, version 2.5, 2015. \\
    \hline
        \centering [Simulateur ICSim] \label{ICSim} & Simulateur d'un tableau de bord de voiture open source, version 3, 2007, \newline \href{https://github.com/zombieCraig/ICSim}{https://github.com/zombieCraig/ICSim}. \\
    \hline
        \centering [PAQL\_B1\_2024] \label{paql} & P. Trémoureux et T. Bénard, Plan d'Assurance Qualité Logicielle, 2023, \path{Git/doc/qualite/PAQL/version/}. \\
    \hline
        \centering [TEST\_B1\_2024] \label{plan_de_test} & P. Trémoureux et T. Bénard, Plan de test, 2023, \path{Git/doc/test/plan_test/livrables/}. \\
    \hline
\end{longtable}