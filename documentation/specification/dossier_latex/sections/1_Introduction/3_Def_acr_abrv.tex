% ---------------------------------------------
% DEFINITION, ACRONYMES ET ABREVIATIONS {p.4,5}
% ---------------------------------------------
\subsection{Définitions, acronymes et abréviations} %1.3
Les abréviations utilisées dans le présent document sont répertoriées et expliquées dans le tableau présenté ci-dessous. Les termes utiles pour interpréter correctement ce dossier de spécifications sont définis dans le dictionnaire de domaine présent dans ce dossier dans la section \ref{dictionnaire}.
% TODO : à compléter au fur et à mesure de la rédaction de ce document
\arrayrulecolor{gray} % couleurs de la structure du tableau en gris pour voir la séparation au milieu de la première ligne
    \begin{longtable}[l]{|>{\centering\arraybackslash} m{0.345\linewidth}|m{0.6\linewidth}|}
        \hline
            \rowcolor{black} % fond en noir pour la première ligne du tableau 
            \textbf{\color{white}Acronymes/Abréviations} & \textbf{\color{white}Définitions} \\
        \hline
            Client \label{client_kereval} & Société KEREVAL, Numéro SIRET 44278921000030. \\
        \hline
            CU & Cas d'utilisation. \\
        \hline
            ID \label{ID} & Identifiant. \\
        \hline
            IEEE (\emph{Institute of Electrical and Electronics Engineers}) & Association professionnelle internationale définissant des normes dans le domaine informatique et électronique. \\
        \hline
            IHM (Interface Homme Machine) & Moyens permettant aux utilisateurs de l'application {\nomApplication} d'interagir avec le programme {\nomLogiciel}. \\
        \hline
            N.A. & Non Applicable. \\
        \hline % utilisation des commandes "\hfill \break" avant "Group" pour éviter les warnings. Cela permet tout simplement un retour à ligne
            OMG (\emph{Object Management Group}) & Consortium international à but non lucratif créé en 1989, dont l'objectif est de standardiser et de promouvoir le modèle objet sous toutes ses formes. \\
        \hline
            SàE (Système à l'Étude) & Ensemble composé de l'application Android, {\nomApplication}, et du programme en C, {\nomLogiciel}. \\
        \hline % ajouter : "\hfill \break" en cas de warning bleu avant "Language"
            UML (\emph{Unified Modeling Language}) & Notation graphique normalisée, définie par l'OMG et utilisée en génie logiciel. \\
        \hline
            TCP/IP \label{tcp_ip} (\emph{Transmission Control Protocol/Internet Protocol}) & Protocole de communication utilisé pour transmettre des données entre l'application {\nomApplication} et le programme {\nomLogiciel}. \\
        \hline
    \end{longtable}
\arrayrulecolor{black}