% ----------------------------------------------
% CU ENVOYER DES TRAMES
% ----------------------------------------------
\newpage
\subsubsection{CU Envoyer des trames}
\paragraph{Description graphique}
Voir figure \ref{schema_cu_strat}.

\paragraph{Description textuelle}
\medskip

\begin{longtable}[l]{|p{3cm}|p{11.7cm}|}
    \hline
    
        Titre & Envoyer des trames.\\
    \hline
    
        Résumé & Utilisateur utilise l'application {\nomApplication} afin d'envoyer une trame CAN au Tableau de Bord. \\
    \hline
    
        Portée & SàE.\\
    \hline
    
        Niveau & Utilisateur\\
    \hline
    
        Acteurs directs & Utilisateur.\\
    \hline 
    
        Acteurs indirects & N.A. \\
    \hline
    
        Préconditions & 
        \begin{itemize}
            \item Le SàE est démarré.
            \item Un ou plusieurs objets existent.
            \item Une ou plusieurs trames existent.
            \item L'application {\nomApplication} est connectée au programme \newline {\nomLogiciel}.
            \item La Raspberry Pi est connectée au bus CAN.
            \item Tableau de Bord est connecté au bus CAN.
        \end{itemize}
        \\
    \hline
    
        Garanties \newline minimales & N.A. \\
    \hline
    
        Garanties en cas de succès & 
        \begin{itemize}
            \item La trame CAN est envoyée à Tableau de Bord.
        \end{itemize}
         \\
    \hline
    
        Scénario nominal & 
        \begin{enumerate}
            \item Utilisateur déroule le menu d'un objet.
            \item L'application {\nomApplication} met à jour EcranPrincipal.
            \item Utilisateur sélectionne une trame.
            \item L'application {\nomApplication} met à jour EcranPrincipal.
            \item Utilisateur demande à envoyer la trame.
            \item Le SàE diffuse la trame sur le bus CAN.
            \item L'application {\nomApplication} met à jour EcranPrincipal.
        \end{enumerate}
        \\
    \hline
        Variantes &     \newline
        \textbf{1 [L'objet est déjà déplié]} \newline
            1.a.1. Va en 3. \newline
            \newline
        \textbf{3 [Une trame est sélectionnée \& Utilisateur ne souhaite pas sélectionner de trames]} \newline
            3.a.1. Va en 5. \newline
            \newline
        \textbf{3 [Aucune trame n'est sélectionnée \& Utilisateur ne souhaite pas sélectionner de trames]} \newline
            3.b.1. Fin du CU. \newline
            \newline
        \textbf{5 [Utilisateur souhaite sélectionner plusieurs trames]} \newline
            5.a.1. Va en 1. \newline
            \newline
        \textbf{5 [Utilisateur ne souhaite plus envoyer de trames]} \newline
            5.b.1. Utilisateur désélectionne les trames grisées. \newline
            5.b.2. L'application {\nomApplication} met à jour EcranPrincipal. \newline
            5.b.3. Fin du CU. \newline
            \newline
        \textbf{5 [Utilisateur souhaite désélectionner une trame]}\newline
            5.c.1. Utilisateur désélectionne une trame grisée. \newline
            5.c.2. L'application {\nomApplication} met à jour EcranPrincipal. \newline
            5.c.3. Va en 3. \newline
            \newline
        \textbf{6 [Une ou plusieurs trames sélectionnées sont paramétrées avec le Mode Envoi cyclique]} \newline
            6.a.1. Le SàE commence l'envoi des trames sélectionnées sur le bus CAN. \newline
            6.a.2. Va en 7. \newline
        \\
    \hline
        Extensions &  N.A. \\
     
    \hline
    Informations \newline complémentaires & N.A. \\
    \hline
\end{longtable}