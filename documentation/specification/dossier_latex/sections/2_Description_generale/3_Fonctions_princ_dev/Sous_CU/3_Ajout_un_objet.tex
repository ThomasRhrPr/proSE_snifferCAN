% ----------------------------------------------
% CU CREER UN NOUVEL OBJET
% ----------------------------------------------
\newpage
\subsubsection{CU Ajouter un objet}
\paragraph{Description graphique}
Voir figure \ref{schema_cu_strat}.
\paragraph{Description textuelle}
\medskip

\begin{longtable}[l]{|p{3cm}|p{11.7cm}|}
    \hline
    
        Titre & Ajouter un objet.\\
    \hline

        Résumé & Utillisateur ajoute un objet sur l'application {\nomApplication}. \\
    \hline

        Portée & Application {\nomApplication}.\\
    \hline

        Niveau & Utilisateur. \\
    \hline

        Acteurs directs & Utilisateur.\\
    \hline 

        Acteurs indirects & N.A. \\
    \hline

        Préconditions & 
        \begin{itemize}
            \item L'application {\nomApplication} est démarrée.
        \end{itemize} \\
    \hline

        Garanties \newline minimales & N.A. \\
    \hline

        Garanties en cas de succès & 
        \begin{itemize}
            \item Un objet est ajouté. 
        \end{itemize}
        \\
    \hline

    Scénario nominal &
    \begin{enumerate} 
        \item Utilisateur demande l'ajout d'un objet. 
        \item Le SàE choisit un Nom d'objet par défaut.
        \item L'application {\nomApplication} affiche PopupAjoutObjet.
        \item Utilisateur saisit un nom d'objet.
        \item Utilisateur valide l'ajout.
        \item L'application {\nomApplication} met à jour EcranPrincipal.
    \end{enumerate} \\
    \hline

        Variantes & \newline
        \textbf{2 [Nombre maximal d'objets atteint \& Utilisateur souhaite ajouter un objet]} \newline
        2.a.1. L’application {\nomApplication} affiche PopupErreurNombreObjet. \newline 
        2.a.2. Utilisateur ferme PopupErreurNombreObjet. \newline
        2.a.3. L’application {\nomApplication} affiche EcranPrincipal. \newline
        2.a.4. Fin du CU. \newline
        \newline
        \textbf{4 [Utilisateur ne souhaite pas saisir de nom]} \newline
        4.a.1. Va en 5. \newline
        \newline
        \textbf{6 [Utilisateur souhaite saisir un nom qui existe déjà]} \newline
        6.a.1. L'application {\nomApplication} affiche PopupErreurAjoutObjet. \newline
        6.a.2. Va en 4. \newline
        \\
    \hline

        Extensions & \newline
        \textbf{4-5 [Utilisateur souhaite annuler]} \newline
        4-5.a.1. Utilisateur ferme PopupAjoutObjet. \newline
        4-5.a.2. L’application {\nomApplication} affiche EcranPrincipal. \newline
        4-5.a.3. Fin du CU. \newline
        \\
    \hline
    Informations \newline complémentaires & N.A. \\
    \hline
\end{longtable}