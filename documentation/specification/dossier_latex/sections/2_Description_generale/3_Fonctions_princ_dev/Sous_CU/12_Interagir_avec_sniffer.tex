% ----------------------------------------------
% CU Interagir avec le sniffer
% ----------------------------------------------
\newpage
\subsubsection{CU Interagir avec le sniffer}
\paragraph{Description graphique}
Voir figure \ref{schema_cu_strat}.
\paragraph{Description textuelle}
\medskip

\begin{longtable}[l]{|p{3cm}|p{11.7cm}|}
    \hline
    
        Titre & Interagir avec le sniffer.\\
    \hline

        Résumé & Utilisateur interagit avec le sniffer. \\
    \hline

        Portée & Application {\nomApplication}.\\
    \hline

        Niveau & Utilisateur.\\
    \hline

        Acteurs directs & Utilisateur.\\
    \hline 

        Acteurs indirects & N.A. \\
    \hline

        Préconditions & 
            \begin{itemize}
                \item Le SàE est démarré. 
                \item L'application {\nomApplication} est connectée au programme \newline {\nomLogiciel}.
                \item La Raspberry Pi est connectée au bus CAN.
                \item Tableau de Bord est connecté au bus CAN.
            \end{itemize}\\
    \hline

        Garanties \newline minimales & N.A. \\
    \hline

        Garanties en cas de succès & 
        \begin{itemize}
            \item Utilisateur est capable d'exporter et de supprimer les trames du sniffer.
            \item Utilisateur est capable d'arrêter la mise à jour du sniffer en temps réel et de revenir en haut du fil lorsqu'il fait défiler les trames.
        \end{itemize}
        \\
    \hline

        Scénario nominal & 
        \begin{enumerate}
            \item Utilisateur demande de revenir en haut du fil.
            \item L'application {\nomApplication} va en haut du fil.
            \item Utilisateur demande d'effacer les trames.
            \item L'application {\nomApplication} efface les trames affichées.
            \item Utilisateur demande l'arrêt de la réception des trames.
            \item L'application {\nomApplication} stoppe l'affichage de nouvelles trames.
            \item Utilisateur demande d'enregistrer les trames reçues sur le \newline Smartphone.
            \item L'application {\nomApplication} enregistre les trames dans un fichier de log (voir section \ref{dictionnaire}).
            \item Utilisateur demande la reprise de la réception des trames.
            \item L'application {\nomApplication} reprend l'affichage de nouvelles trames.
        \end{enumerate} \\
    \hline

        Variantes & \newline
            \textbf{1,3,5,7 [La réception de trames est arrêtée \& Utilisateur souhaite reprendre la réception de trames]}\newline
                1,3,5,7.a.1. Va en 9. \newline
            \newline
            \textbf{1,3,5,7,9 [Utilisateur ne souhaite rien faire]}\newline
                1,3,5,7,9.a.1. Fin du CU. \newline
            \newline
            \textbf{1,3,9 [La réception de trames est arrêtée \& Utilisateur souhaite enregistrer les trames reçues]}\newline
                1,3,9.a.1. Va en 7. \newline
            \newline
            \textbf{1,3,9 [La réception de trames est en cours \& Utilisateur souhaite arrêter la réception de trames]}\newline
                1,3,9.b.1. Va en 5. \newline
            \newline
            \textbf{1,3,9 [La réception de trames est en cours \& Utilisateur souhaite enregistrer les trames reçues]}\newline
                1,3,9.c.1. Va en 5. \newline
            \newline
            \textbf{1,5,7,9 [Utilisateur souhaite effacer les trames]}\newline
                1,5,7,9.a.1. Va en 3. \newline
            \newline
            \textbf{3,5,7,9 [Utilisateur souhaite aller en haut du fil]}\newline
                3,5,7,9.a.1. Va en 1. \newline
            \\
    \hline

        Extensions & N.A. \\
    \hline
    Informations \newline complémentaires & N.A. \\
    \hline
\end{longtable}