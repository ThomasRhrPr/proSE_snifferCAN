% ----------------------------------------------
% CU Recevoir des trames
% ----------------------------------------------
\newpage
\subsubsection{CU Recevoir des trames}
\paragraph{Description graphique}
Voir figure \ref{schema_cu_strat}.
\medskip
\paragraph{Description textuelle}
\medskip

\begin{longtable}[l]{|p{3cm}|p{11.7cm}|}
    \hline
    
        Titre & Recevoir des trames.\\
    \hline

        Résumé & L'application {\nomApplication} reçoit et affiche les trames du réseau CAN. \\
    \hline

        Portée & Application {\nomApplication}.\\
    \hline

        Niveau & Utilisateur.\\
    \hline

        Acteurs directs & Utilisateur.\\
    \hline 

        Acteurs indirects & N.A. \\
    \hline

        Préconditions & 
            \begin{itemize}
                \item Le SàE est démarré. 
                \item L'application {\nomApplication} est connectée au programme \newline {\nomLogiciel}.
                \item La Raspberry Pi est connectée au bus CAN.
                \item Tableau de Bord est connecté au bus CAN.
            \end{itemize}\\
    \hline

        Garanties \newline minimales & N.A. \\
    \hline

        Garanties en cas de succès & 
        \begin{itemize}
            \item Les trames sont affichées sur le sniffer de l'application \newline {\nomApplication}.
        \end{itemize}
        \\
    \hline

        Scénario nominal &
        \begin{enumerate}
            \item Utilisateur commande un actionneur sur Tableau de Bord.
            \item Le SàE récupère les trames du réseau CAN.
            \item L'application {\nomApplication} met à jour EcranPrincipal.
            \item Utilisateur demande à \underline{interagir avec le sniffer}.
        \end{enumerate} \\
    \hline

        Variantes & \newline
            \textbf{1-2 [La connexion échoue \& Utilisateur ne souhaite pas se reconnecter]} \newline
            1-2.a.1. Va en 3. \newline
            \newline
            \textbf{1-2 [La connexion échoue \& Utilisateur souhaite se reconnecter]} \newline
            1-2.b.1. Utilisateur demande à \underline{reconnecter l'application {\nomApplication}} au programme {\nomLogiciel}. \newline
            1-2.b.2. Va en 1. \newline
            \newline
            \textbf{4 [Utilisateur souhaite commander un autre actionneur]}\newline
            4.a.1. Va en 1. \newline
            \\
    \hline

        Extensions & \newline
        \textbf{2-3 [Utilisateur a demandé l'arrêt de la réception de trames]}
        2-3.a.1. Fin du CU. \newline
        \\
    \hline
    Informations \newline complémentaires & N.A. \\
    \hline
\end{longtable}