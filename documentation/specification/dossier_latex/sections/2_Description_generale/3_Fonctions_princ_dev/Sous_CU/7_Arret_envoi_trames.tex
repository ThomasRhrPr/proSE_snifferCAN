% ----------------------------------------------
% CU Arreter l'envoi des trames 
% ----------------------------------------------
\newpage
\subsubsection{CU Arrêter l'envoi des trames}
\paragraph{Description graphique}
Voir figure \ref{schema_cu_strat}.

\paragraph{Description textuelle}
\medskip

\begin{longtable}[l]{|p{3cm}|p{11.7cm}|}
    \hline
    
        Titre & Arrêter l'envoi des trames.\\
    \hline
    
        Résumé & Utilisateur utilise l'application {\nomApplication} afin d'arrêter l'envoi des trames à Tableau de Bord. \\
    \hline
    
        Portée & SàE.\\
    \hline
    
        Niveau & Utilisateur.\\
    \hline
    
        Acteurs directs & Utilisateur.\\
    \hline 
    
        Acteurs indirects & N.A. \\
    \hline
    
        Préconditions & 
        \begin{itemize}
            \item Le SàE est démarré.
            \item L'application {\nomApplication} est connectée au programme \newline {\nomLogiciel}.
            \item La Raspberry Pi est connectée au bus CAN.
            \item Tableau de Bord est connecté au bus CAN.
            \item Un ou plusieurs objets existent.
            \item Une ou plusieurs trames existent.
            \item Une ou plusieurs trames sont en cours d'envoi.
        \end{itemize}
        \\
    \hline
    
        Garanties \newline minimales & N.A. \\
    \hline
    
        Garanties en cas de succès & 
        \begin{itemize}
            \item Plus aucune trame CAN n'est envoyée à Tableau de Bord.
        \end{itemize}
         \\
    \hline
    
        Scénario nominal &
        \begin{enumerate}
            \item Utilisateur demande à stopper l'envoi des trames.
            \item L'application {\nomApplication} affiche PopupArretEnvoi.
            \item Utilisateur confirme l'arrêt de l'envoi des trames.
            \item Le SàE arrête de diffuser les trames sur le bus CAN.
            \item L'application {\nomApplication} affiche EcranPrincipal.
        \end{enumerate}
        \\
    \hline
        Variantes & \newline
        \textbf{3 [Utilisateur ne souhaite plus arrêter l'envoi des trames]} \newline
            3.a.1. Utilisateur ferme le pop-up. \newline
            3.a.2. Va en 5. \newline
        \\
    \hline
        Extensions &  N.A. \\
     
    \hline
    Informations \newline complémentaires & N.A. \\
    \hline
\end{longtable}