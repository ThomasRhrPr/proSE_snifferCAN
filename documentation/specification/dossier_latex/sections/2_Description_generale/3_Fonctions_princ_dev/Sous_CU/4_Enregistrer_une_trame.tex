% ----------------------------------------------
% CU Enregistrer des trames CAN
% ----------------------------------------------
\newpage
\subsubsection{CU Ajouter une trame}
\paragraph{Description graphique}
Voir figure \ref{schema_cu_strat}.
\paragraph{Description textuelle}
\medskip

\begin{longtable}[l]{|p{3cm}|p{11.7cm}|}
    \hline
    
        Titre & Ajouter une trame.\\
    \hline

        Résumé & Utilisateur ajoute une trame CAN associée à un objet avec son Mode Envoi. \\
    \hline

        Portée & Application {\nomApplication}.\\
    \hline

        Niveau & Utilisateur.\\
    \hline

        Acteurs directs & Utilisateur.\\
    \hline 

        Acteurs indirects & N.A. \\
    \hline

        Préconditions & 
        \begin{itemize}
            \item Le SàE est démarré.
            \item Un objet existe.
        \end{itemize} \\
    \hline

        Garanties \newline minimales & N.A.\\
    \hline

        Garanties en cas de succès & 
        \begin{itemize}
            \item Une trame est créée dans le bon objet.
        \end{itemize}
         \\
    \hline

        Scénario nominal &
        \begin{enumerate}
            \item Utilisateur déroule le menu de l'objet.
            \item L'application {\nomApplication} met à jour EcranPrincipal.
            \item Utilisateur saisit la trame. 
            \item Utilisateur valide la saisie de la trame.
            \item L'application {\nomApplication} affiche PopupModeEnvoiTrame.
            \item Utilisateur choisit le mode d'envoi ponctuel.
            \item Utilisateur valide le mode d'envoi.
            \item L'application {\nomApplication} affiche EcranPrincipal.
        \end{enumerate} \\
    \hline

        Variantes & \newline
        \textbf{1 [L’objet est déjà déplié]} \newline
            1.a.1. Va en 3.\newline
            \newline
        \textbf{5 [Nombre maximal de trames atteint \& Utilisateur souhaite
        ajouter une trame]} \newline
            5.a.1. L’application {\nomApplication} affiche PopupErreurNombreTrame.\newline
            5.a.2. Utilisateur ferme PopupErreurNombreTrame. \newline
            5.a.3. L’application {\nomApplication} affiche EcranPrincipal. \newline
            5.a.4. Fin du CU. \newline
            \newline
        \textbf{5 [Utilisateur souhaite ajouter une trame qui n'a pas le bon format]} \newline
            5.b.1. L'application {\nomApplication} affiche PopupErreurSaisieTrame. \newline
            5.b.2. Utilisateur ferme PopupErreurSaisieTrame. \newline
            5.b.3. L’application {\nomApplication} affiche EcranPrincipal. \newline
            5.b.4. Va en 3. \newline 
            \newline 
        \textbf{6 [Utilisateur choisit le mode cyclique \& souhaite changer la périodicité]} \newline
            6.a.1. Utilisateur choisit le mode d'envoi cyclique. \newline
            6.a.2. Utilisateur saisit la périodicité. \newline
            6.a.3. Va en 7. \newline
            \newline
        \textbf{6 [Utilisateur choisit le mode cyclique \& souhaite garder la périodicité par défaut]} \newline
            6.b.1. Utilisateur choisit le mode d'envoi cyclique. \newline    
            6.b.2. Va en 7. \newline
            \\
    \hline

        Extensions & \newline
        \textbf{6-7 [Utilisateur souhaite annuler]} \newline
        6-7.a.1. Utilisateur ferme PopupModeEnvoiTrame. \newline
        6-7.a.2. L'application {\nomApplication} affiche EcranPrincipal. \newline
        \\
    \hline
    Informations \newline complémentaires & N.A. \\
    \hline
\end{longtable}