% ----------------------------------------------
% CU Supprimer un élement 
% ----------------------------------------------
\newpage
\subsubsection{CU Supprimer un élément}
\paragraph{Description graphique}
Voir figure \ref{schema_cu_strat}.

\paragraph{Description textuelle}
\medskip

\begin{longtable}[l]{|p{3cm}|p{11.7cm}|}
    \hline
    
        Titre & Supprimer un élément.\\
    \hline
    
        Résumé & Utilisateur utilise l'application {\nomApplication} pour supprimer une trame ou un objet. \\
    \hline
    
        Portée & SàE.\\
    \hline
    
        Niveau & Utilisateur.\\
    \hline
    
        Acteurs directs & Utilisateur.\\
    \hline 
    
        Acteurs indirects & N.A. \\
    \hline
    
        Préconditions & 
        \begin{itemize}
            \item Le SàE est démarré.
            \item Un ou plusieurs objets existent.
            \item Une ou plusieurs trames existent.
            \item Aucune trame n'est en cours d'envoi.
        \end{itemize}
        \\
    \hline
    
        Garanties \newline minimales & N.A. \\
    \hline
    
        Garanties en cas de succès & 
        \begin{itemize}
            \item L'élément sélectionné est supprimé de l'application. 
        \end{itemize}
        \\
    \hline
    
        Scénario nominal &
        \begin{enumerate}
            \item Utilisateur sélectionne un objet.
            \item L'application {\nomApplication} met à jour EcranPrincipal.
            \item Utilisateur demande la suppression de la sélection.
            \item L'application {\nomApplication} affiche PopupSuppressionElement.
            \item Utilisateur valide la suppression de la sélection.
            \item L'application {\nomApplication} met à jour EcranPrincipal.
        \end{enumerate}
        \\
    \hline
        Variantes &     \newline
        \textbf{1 [Utilisateur souhaite sélectionner une trame]} \newline
            1.a.1. Utilisateur déroule le menu d'un objet. \newline
            1.a.2. Le SàE met à jour EcranPrincipal.\newline
            1.a.3. Utilisateur sélectionne une trame.\newline
            1.a.4. Va en 2.\newline
            \newline
        \textbf{3 [Utilisateur souhaite sélectionner un autre élément]} \newline
            3.a.1. Va en 1.\newline
            \newline
        \textbf{3 [Utilisateur souhaite désélectionner un élément]} \newline
            3.b.1. Utilisateur désélectionne un élément.\newline
            3.b.2. Le SàE met à jour EcranPrincipal.\newline
            3.b.3. Va en 3.\newline
            \newline
        \textbf{5 [Utilisateur ne souhaite plus supprimer d'élément]} \newline
            5.a.1. Utilisateur annule la suppression d'élément. \newline
            5.a.2. Va en 6.\newline
        \\
    \hline
        Extensions &  N.A. \\
     
    \hline
    Informations \newline complémentaires & N.A. \\
    \hline
\end{longtable}