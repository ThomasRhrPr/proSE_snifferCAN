%
% Définition de l'objectif du document
%
\newpage % nouvelle page
\section{Introduction}

\subsection{Objectif du document}
Ce document est un manuel d'utilisation définissant l'ensemble des directives à suivre par le client pour l'utilisation du projet {\guillemotleft} {\projectName} {\guillemotright} du groupe {\teamNumber} {\annee} {\guillemotleft} {\teamName} {\guillemotright} à la suite de sa livraison.\\

Ce document est disponible sur le Référentiel Documentaire Projet (RDP) dans le répertoire [manuel/manuel\_utilisation] sous le nom [Manuel\_Utilisation\_SANS\_B1\_2024].

\subsection{Portée}
Ce document est destiné :
\begin{itemize}
    \item À l'équipe projet ;
    \item Au client KEREVAL.
\end{itemize}
Il pourra être consulté par les consultants de la société FORMATO sur leur demande.

\subsection{Copyright}
Le présent document est un document à but pédagogique. Il a été réalisé sous la direction de Paul TRÉMOUREUX dans le cadre du projet de l'Équipe ProSE B1 2024 CANvengers.\\
Ce document et la propriété de l'Équipe ProSE B1 2024 CANvengers. En dehors des activités pédagogiques de l'ESEO, ce document ne peut être diffusé ou recopié sans l'autorisation écrite de son propriétaire.\\

\subsection{Références}

Voici un tableau récapitulatif des documents utilisés pour le manuel d'utilisation ainsi que les liens permettant d’accéder aux fichiers.
\medskip
% TODO : ajouter toutes les références nécessaires
\begin{longtable}[l]{|m{0.3\linewidth}|m{0.6\linewidth}|} 
    \hline
        \centering [CdC\_KEREVAL\_2023] \label{cdc_kereval} & Société {\client} "Cahier des charges : développement d'une Passerelle Android-CAN vers banc CAN réel ou simulé", 2023. \\
    \hline
        \centering [ISO/IEC/IEEE 29148 : 2018] \label{norme} & International standard, systems and software engineering life cycle processes requirements engineering, 2018, \href{https://standards.ieee.org/standard/29148-2018.html}{https://standards.ieee.org/standard/29148-2018.html}.\\
    \hline
        \centering [UML\_2.5] \label{uml_version} & OMG, Unified Modeling Language, version 2.5, 2015. \\
    \hline
        \centering [Simulateur ICSim] \label{ICSim} & Société {\client}, Simulateur d'un tableau de bord de voiture, version 3, 2007. \\
    \hline
        \centering [PAQL\_B1\_2024] \label{paql} & P. Trémoureux et T. Bénard, Plan d'Assurance Qualité Logicielle, version 0.3.0, 2023, \path{Git/doc/qualite/PAQL/version/}. \\
    \hline
        \centering [plan\_de\_test \_TEST\_B1\_2024] \label{plan_de_test} & P. Trémoureux et T. Bénard, Plan de test, version 0.5.1, 2023, \path{Git/doc/test/plan_test/livrables/}. \\
    \hline
        \centering [dossier\_de\_specifi-cation\_SPEC\_B1\_2024] \label{SPEC} & {\teamName}, Dossier de specification, version 1.0, 2023, \path{Git/doc/specification/livrables/}. \\
    \hline
        \centering [Manuel\_Installation\_SANS\_B1\_2024] \label{INST} &  {\teamName}, Manuel d'installation, version 1.0, 2023, \path{Git/doc/manuel/livrables/}\\
    \hline
\end{longtable}