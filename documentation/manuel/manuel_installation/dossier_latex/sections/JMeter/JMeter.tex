\section{Installation de JMeter}

Pour la réalisation des tests fonctionnels du serveur TCP implémenté dans le module Postman, nous avons utilisé JMeter.\\

JMeter est un outil open-source de test de performance et de charge développé par Apache. Il permet de mesurer et d'évaluer les performances d'une application web ou d'un serveur en simulant différents scénarios d'utilisation. JMeter offre une grande flexibilité pour la création de tests et fournit des statistiques détaillées sur les performances, la fiabilité et la capacité de l'application testée.\\

Tout d'abord, assurez-vous d'avoir une version récente de Java installée sur votre système, car JMeter est basé sur Java. Pour l'installer, rendez-vous sur le site officiel d'Oracle Java à l'adresse suivante : \href{https://www.oracle.com/java/technologies/javase-jdk11-downloads.html}{https://www.oracle.com/java/technologies/javase-jdk11-downloads.html}.
Dans notre cas nous avons installé le JDK (Java Development Kit) d'OpenJDK 11.\\

Pour installer JMeter il faut :
\begin{itemize}
    \item Allez sur le site officiel d'Apache JMeter à l'adresse suivante : \href{https://jmeter.apache.org/}{https://jmeter.apache.org/}.
    \item Dans la section "Downloads", cliquez sur le lien correspondant à la version la plus récente de JMeter. Vous serez redirigé vers la page de téléchargement.
    \item Sur la page de téléchargement, choisissez le fichier binaire qui correspond à votre système d'exploitation (Windows, Linux, Mac OS, etc.) et téléchargez-le sur votre ordinateur.
    \item Une fois le téléchargement terminé, décompressez le dossier dans le répertoire de votre choix.
    \item Accédez au répertoire extrait, allez dans le répertoire bin/ et lancez JMeter en exécutant le fichier "jmeter.bat" pour Windows ou "jmeter.sh" pour Linux/Mac OS.
    \item JMeter s'ouvrira avec une interface graphique. Vous êtes maintenant prêt à commencer à créer et exécuter des tests de performance.
\end{itemize}