\section{Installation de l'application CANdoid}

CANdroid est lancé depuis Android Studio. Pour cela, nous vous proposons un tutoriel d'installation.

\subsection{Installation de Android Studio}

Rendez-vous sur le site officiel : \href{https://developer.android.com/studio}{Android Studio}  et téléchargez le fichier .zip. Placez le dans le dossier pertinent pour vos applications (par exemple dans /usr/local/). 
Dans un terminal, rendez-vous dans le répertoire android-studio/bin/. Ensuite, exécutez studio.sh.

Choisissez, si vous le souhaitez, d'importer les paramètres anciens de Android Studio. Cliquez sur OK pour valider.
Poursuivez l'installation avec l'assistant de configuration d'Android Studio. Cela implique le téléchargement des composants SDK Android nécessaires au dévéloppement. 

Si vous possédez une version 64 bits d'Ubuntu, il est nécessaire d'installer des bibliothèques 32 bits avec la commande suivante : 
\vspace{-1.8\baselineskip} 
\begin{lstlisting}
    sudo apt install libc6:i386 libncurses5:i386 libstdc++6:i386 lib32z1 libbz2-1.0:i386
\end{lstlisting}

Si vous exécutez Fedora 64 bits, tapez : 
\vspace{-1.8\baselineskip} 
\begin{lstlisting}
    sudo yum install zlib.i686 ncurses-libs.i686 bzip2-libs.i686
\end{lstlisting}

\subsection{Configurer l'appareil}

Sur votre Smartphone, dans les paramètres, activez les option pour les développeurs. 
Appuyez sept fois sur l'option Build Number (Numéro de version) de manière à ce que le message "You are now a developer!" s'affiche. Cela permet d'activer les options pour les développeurs sur votre appareil. 

Vous devez ensuite activer le débogage USB de manière à ce que les outils SDK reconnaissent l'appareil s'il est connecté via USB. 
Activez le débogage USB dans les paramètres système de l'appareil dans Developer options (Options pour les développeurs). Le chemin d'accès dépend de la version Android : 

\begin{itemize}
    \item Android 9 (niveau d'API 28) ou version ultérieure : Settings > System > Advanced > Developer Options > USB debugging (Paramètres > Système > Avancé > Options pour les développeurs > Débogage USB) 
    \item Android 8.0.0 (niveau d'API 26) et Android 8.1.0 (niveau d'API 27) : Settings > System > Developer Options > USB debugging (Paramètres > Système > Options pour les développeurs > Débogage USB) 
    \item Android 7.1 (niveau d'API 25) ou version antérieure : Settings > Developer Options > USB debugging (Paramètres > Options pour les développeurs > Débogage USB)
\end{itemize}



Lorsque Android Studio est fonctionnel, branchez votre appareil. Dans Open'Edit Run/Debug configurations' Dialog > Edit Configuration... > CANdroid > Apply > OK. Puis cliquez sur Run 'CANdroid'. 