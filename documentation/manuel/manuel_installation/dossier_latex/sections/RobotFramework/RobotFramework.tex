\section{Installation de Robot Framework}

Afin d'exécuter des tests automatisés, nous vous proposons un tutoriel d'installation de PyCharm et Robot Framework.

\subsection{Installation de PyCharm}

Commencez par installer python, dans un terminal tapez : 
\vspace{-1.8\baselineskip} 
\begin{lstlisting}
    sudo apt install python3
    sudo apt install python3-pip
\end{lstlisting}

Téléchargez l'archive de PyCharm sur le site officiel : \href{https://www.jetbrains.com/fr-fr/pycharm/download/other.html}{Pycham}. Prenez la version 2022.3.3 (Community edition) pour votre OS. Désarchivez l'archive dans le dossier de votre choix.

Téléchargez appium :
\vspace{-1.8\baselineskip} 
\begin{lstlisting}
    sudo apt install npm
    npm install appium
\end{lstlisting}

Téléchargez Robot Framework :
\vspace{-1.8\baselineskip} 
\begin{lstlisting}
    pip install robotframework
    pip install robotframework-appiumlibrar
    pip install --force-reinstall -v "Appium-Python-Client==2.8.1"
    pip install --force-reinstall -v "selenium==4.8.2"
\end{lstlisting}

\subsection{Configuration de PyCharm}

Commencez par lancer PyCharm : 
\vspace{-1.8\baselineskip} 
\begin{lstlisting}
    ./<path>/<to>/pycharm-community-2022.3.3/bin/pycharm.sh
\end{lstlisting}

Allez dans le menu File > Settings > Plugins, tapez "robot", installez "Robot Runner" et "IntelliBot \#patched". Redémarrez l'IDE.